《本草綱目》原序

紀稱:望龍光,知古劍;覘寶氣,辨明珠。故萍實商羊,非天明莫洞。厥後博物稱\ul{華},辨字稱\ul{康},析寶玉稱\ul{倚頓},亦僅僅晨星耳。\ul{楚}\ul{蘄陽}\ul{李君}\ul{東璧},一日過予弇山園謁予,留飲數有《本草綱目》數十卷。謂予曰:「\ul{時珍},\ul{荆楚}鄙人也。幼多羸疾,質成鈍椎;長耽典籍,若啖蔗飴。遂漁獵群書,蒐羅百氏。凡子、史、經、傳、聲韻、農圃、醫卜、星相、樂府諸家,稍有得處,輒著數言。古有《本草》一書,自\ul{炎}、\ul{黃}及\ul{漢}、\ul{梁}、\ul{唐}、\ul{宋},下迨國朝,註解群氏舊矣。第其中舛謬差訛遺漏,不可枚數。乃敢奮編摩之志,僭纂述之權。歲歷三十稔,書考八百餘家,稿凡三易。復者芟之,闕者緝之,訛者繩之。舊本一千五百一十八種,今增藥三百七十四種,分爲一十六部,著成五十二卷。雖非集成,亦粗大備,僭名曰《本草綱目》。願乞一言,以托不朽。予開卷細玩,每藥標正名爲綱,附釋名爲目,正始也。次以集解、辨疑、正誤,詳其土産形狀也。次以氣味、主治、附方,著其體用也。上自墳典,下及傳奇,凡有相關,靡不備採。如入\ul{金谷}之園,種色奪目;如登龍君之宮,寶藏悉陳;如對冰壺玉鑒,毛髮可指數也。博而不繁,詳而有要,綜核究竟,直窺淵海。茲豈僅以醫書覯哉!實性理之精微,格物之通典,帝王之秘,臣民之重寶也。\ul{李君}用心嘉惠何勤哉!噫,玉莫剖,朱紫相傾,弊也久矣。故辨專車之骨,必俟\ul{魯}儒;博支機之石,必訪賣卜。予方著《弇州卮言》,恚博古如《丹鉛》、《卮言》後乏人也。何幸睹茲集哉!茲集也,藏之深山石室無當,盍鍥之,以共天下後世味《太玄》如\ul{子雲}者。時\ul{萬曆}歲庚寅春上元日,\ul{弇州山人}\ul{鳳洲}\ul{王世貞}拜撰。