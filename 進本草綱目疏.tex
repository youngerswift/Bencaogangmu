
進本草綱目疏

\ul{湖廣}\ul{黃州府}儒學增廣生員\ul{李建元}謹奏,爲遵奉明例訪書,進獻《本草》,以備採擇事。臣伏讀禮部儀制司勘合一款,恭請聖明敕儒臣開書局纂修正史,移文中外。凡名家著述,有關國家典章,及紀君臣事跡,他如天文、樂律、醫術、方技諸書,但成一家名言,可以垂於方來者,即訪求解送,以備採入《藝文志》。如已刻行者,即刷印一部送部。或其家自欲進獻者,聽。奉此。臣故父\ul{李時珍},原任楚府奉祠,奉敕進封文林郎、\ul{四川}\ul{蓬溪}知縣。生平篤學,刻意纂修。曾著《本草》一部,甫及刻成,忽值數盡,撰有遺表,令臣代獻。臣切思之:父有遺命而子不遵,何以承先志!父有遺書而子不獻,何以應朝命!矧今修史之時,又值取書之會,臣不揣謭陋,不避斧鉞,謹述故父遺表。臣父\ul{時珍},幼多羸疾,長成鈍椎,耽嗜典籍,若啖蔗飴。考古證今,奮發編摩,苦志辨疑訂誤,留心纂述諸書。伏念《本草》一書,關係頗重,註解群氏,謬誤亦多。行年三十,力肆校讎;歷歲七旬,功始成就。野人炙背食芹,尚欲獻之天子;微臣採珠聚玉,敢不上之明君!昔\ul{炎}\ul{黃}辨百穀,嘗百草,而分別氣味之良毒;\ul{軒轅}師\ul{岐伯},遵\ul{伯高},而剖析經絡之本標。遂有《神農本草》三卷,《藝文》錄爲醫家一經。及\ul{漢}末而\ul{李當之}始加校修,至\ul{梁}末而\ul{陶弘景}益以注釋,古藥三百六十五種,以應重卦。\ul{唐高宗}命司空\ul{李}重修,長史\ul{蘇恭}表請伏定,增藥一百一十四種。\ul{宋太祖}命醫官\ul{劉翰}詳校,\ul{宋仁宗}再詔補注,增藥一百種。召醫\ul{唐慎微}合爲《證類》,修補眾本草五百種。自是人皆指爲全書,醫則目爲奧典。夷考其間,?瑕不少。有當析而混者,如葳蕤、女葳,二物而併入一條;有當並而析者,如南星、虎掌,一物而分爲二種;生薑、薯蕷,菜也,而列草品;檳榔、龍眼,果也,而列木部。八谷,生民之天也,不能明辨其種類;三菘,日用之蔬也,罔克的別其名稱。黑豆、赤菽,大小同條;硝石、芒硝,水火混注。以蘭花爲蘭草,卷丹爲百合,此\ul{寇氏}《衍義》之舛謬;謂黃精即鈎吻,旋花即山姜,乃\ul{陶氏}《別錄》之差訛。酸漿、苦耽,草菜重出,\ul{掌氏}之不審;天花、栝蔞,兩處圖形,\ul{蘇氏}之欠明。五倍子,構蟲窠也,而認爲木實;大草,田字草也,而指爲浮萍。似茲之類,不可枚陳,略摘一二,以見錯誤。若不類分品列,何以印定群疑?臣不揣猥愚,僭肆刪述,重復者芟之,遺缺者補之。如磨刀水、潦水、桑柴火、艾火、鎖陽、山柰、土茯苓、番木鱉、金柑、樟腦、蠍虎、狗蠅、白蠟、水蛇、狗寶、秋蟲之類,並今方所用,而古本則無;三七、地羅、九仙子、蜘蛛香、豬腰子、勾金皮之類,皆方物土苴,而稗官不載。今增新藥,凡三百七十四種,類析舊本,分爲一十六部。雖非集成,實亦粗備。有數名或散見各部,總標正名爲綱,余各附釋爲目,正始也;次以集解、辨疑、正誤,詳其出産、形狀也;次以氣味、主治、附方,著其體用也。上自墳典,下至傳奇,凡有相關,靡不收採,雖命醫書,實該物理。我\ul{太祖高皇帝}首設醫院,重設醫學,沛仁心仁術於九有之中;\ul{世宗肅皇帝}既刻《醫方選要》,又刻《衛生易簡》,藹仁政仁聲於率土之遠。伏願皇帝陛下體道守成,遵祖繼志;當離明之正位,司考文之大權。留情民瘼,再修司命之書;特詔良臣,著成昭代之典。治身以治天下,書當與日月爭光;壽國以壽萬民,臣不與草木同朽。臣不勝冀望屏營之至。臣建元爲此一得之愚,上乾九重之覽,或准行禮部轉發史館採擇,或行醫院重修,父子銜恩,存歿均戴。臣無任瞻天仰聖之至。\ul{萬曆}二十四年十一月日進呈,十八日奉聖旨:書留覽,禮部知道。欽此。