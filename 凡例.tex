《本草綱目》凡例

一、《神農本草》三卷,三百六十種,分上、中、下三品。\ul{梁}\ul{陶弘景}增藥一倍,隨品附入。\ul{唐}、\ul{宋}重修,各有增附,或並或退,品目雖存,舊額淆混,義意俱失。今通列一十六部爲綱,六十類爲目,各以類從。三品書名,俱注各藥之下,一覽可知,免尋索也。
一、舊本玉、石、水、土混同,諸蟲、鱗、介不別,或蟲入木部,或木入草部。今各列爲部,首以水、火,次之以土,水、火爲萬物之先,土爲萬物母也。次之以金、石,從土也。次之以草、谷、菜、果、木,從微至巨也。次之以服、器,從草、木也。次之以蟲、鱗、介、禽、獸,終之以人,從賤至貴也。
一、藥有數名,今古不同。但標正名爲綱,余皆附於釋名之下,正始也。仍注各本草名目,紀原也。
一、\ul{唐}、\ul{宋}增入藥品,或一物再出、三出,或二物、三物混注,今俱考正,分別歸並,但標其綱,而附列其目。如標龍爲綱,而齒、角、骨、腦、胎、涎皆列爲目;標粱爲綱,而赤、黃粱米皆列爲目之類。
一、諸品首以釋名,正名也;次以集解,解其出産、形狀、採取也;次以辨疑、正誤,辨其可疑,正其謬誤也;次以修治,謹炮炙也;次以氣味,明性也;次以主治,錄功也;次以發明,疏義也;次以附方,著用也;或欲去方,是有體無用矣。(舊本附方二千九百三十五,今增八千一百六十一。)
一、\ul{唐}、\ul{宋}以朱墨圈蓋分別古今,經久訛謬。今既板刻,但直書諸家《本草》名目於藥名、主治之下,便覽也。
一、諸家《本草》,重復者刪去,疑誤者辨正,採其精粹,各以人名,書於諸款之下,不沒其實,且是非有歸也。
一、諸物有相類而無功用宜參考者,或有功用而人卒未識者,俱附錄之。無可附者,附於各部之末。蓋有隱於古而顯於今者,如莎根即香附子,\ul{陶氏}不識而今則盛行;辟虺雷,昔人罕言而今充方物之類,雖冷僻,不可遺也。
一、\ul{唐}、\ul{宋}本所無,\ul{金}、\ul{元}、我\ul{明}諸醫所用者,增入三十九種。\ul{時珍}續補三百七十四種。雖曰醫家藥品,其考釋性理,實吾儒格物之學,可裨《爾雅》、《詩疏》之缺。
一、舊本序例重繁,今止取《神農》爲正,而旁採《別錄》諸家附於下,益以\ul{張}、\ul{李}諸家用藥之例。
一、古本百病主治藥,略而不切。\ul{王氏}《集要》、\ul{祝氏}《證治》,亦約而不純。今分病原列之,以便施用,雖繁不紊也。
一、《神農》舊目及\ul{宋}本《總目》,附於例後,存古也。